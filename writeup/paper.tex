%-----------------------------------------------------------------------------
%
%               Template for sigplanconf LaTeX Class
%
% Name:         sigplanconf-template.tex
%
% Purpose:      A template for sigplanconf.cls, which is a LaTeX 2e class
%               file for SIGPLAN conference proceedings.
%
% Guide:        Refer to "Author's Guide to the ACM SIGPLAN Class,"
%               sigplanconf-guide.pdf
%
% Author:       Paul C. Anagnostopoulos
%               Windfall Software
%               978 371-2316
%               paul@windfall.com
%
% Created:      15 February 2005
%
%-----------------------------------------------------------------------------


\documentclass[preprint]{sigplanconf}

% The following \documentclass options may be useful:
%
% 10pt          To set in 10-point type instead of 9-point.
% 11pt          To set in 11-point type instead of 9-point.
% authoryear    To obtain author/year citation style instead of numeric.

\usepackage{amsmath, listings, bcprules, amsthm, amssymb, proof}
\usepackage{textcomp}

\begin{document}

\newtheorem{theorem}{Theorem}[section]
\newtheorem{lemma}[theorem]{Lemma}
\newtheorem{definition}[theorem]{Definition}
\newtheorem{proposition}[theorem]{Proposition}

\conferenceinfo{B629 '15}{Bloomington, IN.}
\copyrightyear{2015} 
\copyrightdata{[to be supplied]} 

\lstnewenvironment{code}[1][]%
  {\noindent\medskip 
     %\lstset{basicstyle=\sffamily\footnotesize,frame=single,#1}}
     \lstset{basicstyle=\sffamily\footnotesize,#1}}
  {}

%\titlebanner{banner above paper title}        % These are ignored unless
%\preprintfooter{short description of paper}   % 'preprint' option specified.

\title{Laurens de Graaf}
\subtitle{A Spineless, ``Tagless'' Graph-Reduction Machine in RPython}

%% \title{Semantics for Explicit Contract Monitoring and Support for Multiple Evaluation Strategies}
%\subtitle{Subtitle Text, if any}

\lstset{language=Python, basicstyle=\sffamily}

%% \authorinfo{Name1}
%%            {Affiliation1}
%%            {Email1}
\authorinfo{Cameron Swords}
           {Indiana University}
           {\{cswords\}@indiana.edu}
\maketitle

\begin{abstract}
\begin{center}
\emph{``A Great and Mischievous Pirate''
{\begin{flushright}$\sim$ Henry Morgan\end{flushright}}}
\end{center}
\end{abstract}

%% \category{CR-number}{subcategory}{third-level}
\terms
Languages, JIT, Tracing
\keywords
piracy, rpython, tracing, jit

\section{\emph{Laurencillo}~-- Introduction} %% Introduction


There has been tremendous work into JITs and JIT optimizations in recent years.
Most notably, the RPython tracing JIT has come to be a de-facto standard for the
power of a tracing JIT, driving their flagship python interpreter PyPy. In this
work we present an implementation of the Spinless, Tagless G-Machine (herein the
STG Machine) as an interpreter written on top of RPython. 

The original aim of this work was to construct a fast implementation of the
STG machine that might support a reasonable subset of the compilation output
of GHC. To this end, our implementation is faithful to the original semantics
presented by \citet{spj:stgmachine}.

\section{\emph{Tigre}~-- The STG Machine} %% The Spineless, Tagless G-Machine

In this section, we give a brief sketch of the implementation approach we take
to construct the STG Machine in RPython, discussing the general structure of the
implementation. To begin, we sketch the directory structure and describe each file
in Figure~\ref{fig:dir} to outline the overall project.

\begin{figure}
\begin{itemize}
\item \lstinline{laurens/} -- Top-level directory
  \begin{itemize}
    \item \lstinline{config.py} -- Configuration file (currently unused)
    \item \lstinline{debug.py}  -- Debug utilities, primarily for printing
                                   machine states during execution
    \item \lstinline{op.py}     -- The operation nodes, which each contain
                                   a \lstinline{step} operation that performs
                                   a machine reduction.
    \item \lstinline{parse.py}  -- Input parser (currently unstarted)
    \item \lstinline{stg.py}    -- The main loop of the implementation
    \item \lstinline{tests.py}  -- A set of tests written in terms of AST
                                   nodes for the machine; this serves as
                                   an alternative entry point
  \item \lstinline{laurens/ast/}
    \begin{itemize}
      \item \lstinline{ast.py}  -- AST nodes that make up expressions;
                                   part of \lstinline{op.py} should likely be
                                   refactored into this file
      \item \lstinline{cont.py} -- Continuations for the  Return Stack 
    \end{itemize}
  \item \lstinline{laurens/data/}
    \begin{itemize}
      \item \lstinline{argstack.py} -- Stack for arguments to functions
      \item \lstinline{closure.py}  -- Data structure to represent closures
      \item \lstinline{config.py}   -- Machine Configuration
      \item \lstinline{heap.py}     -- Heap structure
      \item \lstinline{retstack.py} -- Stack for return continuations
      \item \lstinline{updframe.py} -- Update frame tuple
      \item \lstinline{updstack.py} -- Stack for update frames
    \end{itemize}
  \end{itemize}
\end{itemize}
\caption{\label{fig:dir}The directory structure of Laurens.}
\end{figure}

\subsection{The Machine Configuration}

Before we discuss the implementation details, we describe the
STG Machine. This abstract machine uses a six-tuple configuration:
\[\langle op; as; rs; us; h; \sigma\rangle\]
Unsurprisingly, each component plays a unique role in the machine:
\begin{itemize}
  \item $op$ contains the operation for the reduction to perform;
  \item $as$ contains the argument stack, which maintains arguments
        during applications;
  \item $rs$ contains the return stack, which manages return points
        (primarily ``case'' expressions);
  \item $us$ contain the update stack, used for handling thunk updates;
  \item $h$ contains the heap, where memory addresses map to closures;
  \item $\sigma$ contains the \emph{global environment} of bindings.
\end{itemize}
While \citet{spj:stgmachine} observe that many of these stacks may
be combined during compilation to C as an optimization, we maintain
this configuration explicitly in our implementation:
\begin{code}
class Config(object):
  def __init__(self, code, args, ret,  
                     upd, heap, genv):
    self.code       = code
    self.arg_stack  = args
    self.ret_stack  = ret
    self.upd_stack  = upd
    self.heap       = heap
    self.global_env = genv
\end{code}
We take this approach for two reasons. First, we are working from the
assumption that RPython should be able to properly handle the interpretation
structure of the STG machine, and thus these individual stacks should not
cause critical, performance-inhibiting overhead.
Second, RPython performs immense amounts of type inference and so we use
separate stack implementations, as we discuss further in
Section~\ref{sec:typeinf}.

\subsection{The Main Loop}

\begin{figure}
\begin{code}
def terminateHuh(config):
  return 
    (((config.code.op == op.ReturnConOp) or 
      (config.code.op == op.ReturnIntOp)) 
     and config.ret_stack.empty()
     and config.upd_stack.empty())

def loop(config):
  while not terminateHuh(config):
    jitdriver.jit_merge_point(code = 
                                config.code)
    config = config.code.step(config)
  return config
\end{code}
\caption{\label{fig:loop}The main loop and termination test.}
\end{figure}

The \lstinline{stg.py} file serves as the main loop, driven by operation steps
in \lstinline{op.py}. Everything else supports this main interaction.  
Our overall computational mechanism is presented in Figure~\ref{fig:loop}. The
main file contains a loop driver that steps the configuration until there is no
more to do. At each step, we call \lstinline{step} on the current configuration,
receiving a new one. This computation proceeds until we terminate with either
an integer or a constructor and both the update and return stacks are empty.

The operations are broken up by the STG machine into four possible operations:
evaluation, entering, returning a construct, and returning an integer, given
as \emph{Eval $e$ $\rho$}, \emph{Enter $a$}, \emph{ReturnInt $i$}, and
\emph{ReturnCon $c$ $xs$} in the reduction semantics. We implement each of these
as an \emph{operation node} in \lstinline{op.py}, wherein each has a
\lstinline{step} method that expects a configuration using that operator and
dispatches appropriately, returning the new machine configuration.

\subsection{Operations}

There are four operations in the STG machine which indicate the next
step the machine will take. We implement each of these as a \lstinline{step}
attached to an operation node in \lstinline{op.py}, allowing us to easily
modularize the main loop.

\paragraph{\textit{Eval e $\rho$}.}
The \textit{Eval} step evaluates an operation, yielding either a 
sub-expression to evaluate (in the case of a \lstinline{case} operation)
or an \textit{Enter} operation with a heap address location (such as when
evaluating an application).
This is, by far, the most grotesque part of the implementation: to implement the
evaluation step, we extract the expression node and branch on its type,
constructing a new configuration as necessary based on the result.

\paragraph{\textit{Enter a}.}
The \textit{Enter} step enters a heap address, which is a \emph{closure} stored
in the heap. This operation also handles thunk updates and partial applications,
constructing an updated thunk in the former case and a new thunk with additional
input in the latter.  

\paragraph{\textit{ReturnInt i}.}
This operation indicates that an expression has evaluated to an integer $i$. The
return stack will contain a \lstinline{case} expression to match against this
result. We present this case in full in Figure~\ref{fig:ReturnInt}; the 
\lstinline{int_case_lookup} operator takes the relevant cases and identifies the
appropriate one. We also indicate a \emph{default case}, which uses a binder
with a default variable to bind the result if it does not match an integer
literal.

\begin{figure}
\begin{code}
def step(self,config):
    debug("Int return")

    if config.ret_stack.empty():
      raise Exception('...')
    else: 
      retk      = config.ret_stack.pop()
      if type(retk) is ast.cont.CaseCont:
        ret_env   = retk.env.copy()
        ret_alts  = retk.alts
        value     = self.value

        default,default_var,body = 
          self.int_case_lookup(value, ret_alts)

        if default:
          if (default_var is not None):
            var = default_var
            ret_env[var] = 
              ast.ast.Value(value, True)

        config.code = op.Eval(body, ret_env)
  
    return config

  def int_case_lookup(self,val,ret_alts):
    # Assuming everything is a LitAlt
    for alt in ret_alts.alternates: 
      if alt.literal == val:
        return (False, None, alt.rhs)

    return (True, 
            ret_alts.default.binder, 
            ret_alts.default.rhs)
\end{code}
\caption{\label{fig:returnint}The \textit{ReturnInt} operation.}
\end{figure}

\paragraph{\textit{ReturnCon c xs}.}
This operation indicates that an expression has evaluated to a constructor $c$
with arguments in the $xs$. The return stack (or, if it is empty, the update
stack) will contain a \lstinline{case} expression to match against this result.

\section{\emph{Francesca}~-- RPython Usage \& Notes} %% RPython

In this section, we discuss a number of intricacies that require careful
consideration during implementation of the STG machine before describing
how we have leveraged the \lstinline{jitdriver} in our implementation.

\subsection{RPython Customizations}



\paragraph{Unique Stack Implementations.}\label{sec:typeinf}
As previously state, RPython attempts exhaustive type inference for structures
during compilation. This approach has forced us into using three different
stack implementations, all identical in code; we would like to use a single
stack defined as \lstinline{data/stack.py}, but these authors could not find
a way to circumvent the type inference to allow such reuse.
Indeed, we would suggest that this is one of the largest
hurdles in constructing a project such as this in RPython: there is a non-zero
amount of code duplication required to simply exist in the RPython ecosystem.

Furthermore, the machine description~\cite{spj:stgmachine} describe an
optimization for combining stacks that would suggest that we can combine these
stacks by carefully packaging each different node in a new object to indicate
which of the original stacks it lived on\footnote{There are other mechanisms to
avoid this discussed in the original work, but they require subtle bit
manipulation that would need to be faked in a higher-level implementation.}. 
Unfortunately, this would add one further level of indirection for the compiler
and further obfuscate our implementation, and it is unclear if we could gain
speed for the effort.  Further experimentation should be done, here, however.  

\paragraph{The Funny Case of Laziness.}
While standard lazy languages provide forcing at primitive operations, the
STG machine departs from this explicit approach. Instead, a primitive operation
assumes that its arguments are pre-evaluated, and this became troublesome when
testing them. To further elaborate this point, observe a standard definition
of addition in the STG language:
\begin{code}[language=Haskell]
+ = \ {} \\n {n,m} -> case n of
                        n' -> case m of
                                m' -> #+ n' m'
\end{code}
This snippet is interpreted as: 
\begin{center}
\emph{
``Plus is a lambda with no free variables that
is should not be updated. When applied, it will \lstinline{case} on
\lstinline{n} and then \lstinline{m}, using the forcing position of
\lstinline{case} to ensure that both inputs are evaluated before using the
primitive addition operator \lstinline{\#+} on the result.''
}
\end{center}
While not particularly tied to the implementation in RPython, we found this
insight invaluable during implementation and testing and so we have included
it here. This manifests with the following dispatch for addition in our
implementation:
\begin{code}
elif expr_type is ast.ast.PrimOp:
  if cexp.oper == "+":
    lookups = vals(code.env,
                   global_env, 
                   cexp.atoms)
    x1      = lookups[0]
    x2      = lookups[1]
    assert isinstance(x1,ast.ast.ValAST)
    assert isinstance(x2,ast.ast.ValAST)
    res     = x1.value + x2.value
    config.code = op.ReturnInt(res) 
\end{code}
We first ensure that we are performing addition and then look up the operators
with \lstinline{vals} (which uses the local and global environments for variable
lookup). Next, we ensure that each of the resulting lookups are valid Value
nodes, extract their values, and then construct a \lstinline{ReturnIn} operation
code for the configuration.

\subsection{The RPython JIT Driver.}

\begin{code}
from rpython.rlib.jit import JitDriver, purefunction

def get_location(code):
    return "%s" % ( code )

jitdriver = JitDriver(greens=['code']
                     , reds='auto'
                     , get_printable_location=get_location)

def terminateHuh(config):
  return (((config.code.op == op.ReturnConOp) or 
           (config.code.op == op.ReturnIntOp)) 
          and config.ret_stack.empty()
          and config.upd_stack.empty())

def loop(config):

  while not terminateHuh(config):
    jitdriver.jit_merge_point( code       = config.code)
    config = config.code.step(config)
  return config
\end{code}

\section{\emph{Neptune} n\'{e}e \emph{Fortune}~-- Conclusion} %% Conclusion / Future Work

Finally, we discuss our timing results, and must report a negative one:
the final implementation is \emph{slower} when we perform tracing on
a sample program computing the 30th Fibonacci number. This is almost certainly
the result of our na\"{i}ve tracing mechanism, wherien we consider every node as
a possible merge point for the JIT, introducing immense tracing overhead to the
computation. Thus it is unclear if Tracing is suitable for the STG machine
because we have not performed effort to fine-tune our implementation.

Beyond this result, we still have much to do:
\begin{itemize}
\item Write a full parser that will construct AST nodes from reasonable input;
\item Write a more extensive test suite to examine and use constructors,
      \lstinline{let}, \lstinline{letrec}, laziness, etc.; this will be much
      easier once a parser is written;
\item Support a full range of built-in functions, including I/O and additional
      numeric operators;
\item Replace the dictionary-based representation of environments with a
      more efficient data structure
\item Fine-tune the JIT heuristics to gain speed.
\end{itemize}

%% \section{Appendix Title}
%% 
%% This is the text of the appendix, if you need one.

%% \acks
%% 
%% Acknowledgments, if needed.

% We recommend abbrvnat bibliography style.

\bibliographystyle{abbrvnat}

% The bibliography should be embedded for final submission.

\softraggedright
\bibliography{p}

%%\appendix
%%\input{proof-appendix.tex}

\end{document}
