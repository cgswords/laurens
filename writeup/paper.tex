%-----------------------------------------------------------------------------
%
%               Template for sigplanconf LaTeX Class
%
% Name:         sigplanconf-template.tex
%
% Purpose:      A template for sigplanconf.cls, which is a LaTeX 2e class
%               file for SIGPLAN conference proceedings.
%
% Guide:        Refer to "Author's Guide to the ACM SIGPLAN Class,"
%               sigplanconf-guide.pdf
%
% Author:       Paul C. Anagnostopoulos
%               Windfall Software
%               978 371-2316
%               paul@windfall.com
%
% Created:      15 February 2005
%
%-----------------------------------------------------------------------------


\documentclass[preprint]{sigplanconf}

% The following \documentclass options may be useful:
%
% 10pt          To set in 10-point type instead of 9-point.
% 11pt          To set in 11-point type instead of 9-point.
% authoryear    To obtain author/year citation style instead of numeric.

\usepackage{amsmath, listings, bcprules, amsthm, amssymb, proof}
\usepackage{textcomp}

\begin{document}

\newtheorem{theorem}{Theorem}[section]
\newtheorem{lemma}[theorem]{Lemma}
\newtheorem{definition}[theorem]{Definition}
\newtheorem{proposition}[theorem]{Proposition}

\conferenceinfo{B629 '15}{Bloomington, IN.}
\copyrightyear{2015} 
\copyrightdata{[to be supplied]} 

\lstnewenvironment{code}[1][]%
  {\noindent\medskip 
     \lstset{basicstyle=\sffamily\footnotesize,frame=single,#1}}
  {}

%\titlebanner{banner above paper title}        % These are ignored unless
%\preprintfooter{short description of paper}   % 'preprint' option specified.

\title{Laurens de Graaf}
\subtitle{A Spineless, ``Tagless'' Graph-Reduction Machine in RPython}

%% \title{Semantics for Explicit Contract Monitoring and Support for Multiple Evaluation Strategies}
%\subtitle{Subtitle Text, if any}

\lstset{language=Python, basicstyle=\sffamily}

%% \authorinfo{Name1}
%%            {Affiliation1}
%%            {Email1}
\authorinfo{Cameron Swords}
           {Indiana University}
           {\{cswords\}@indiana.edu}
\maketitle

\begin{abstract}
\begin{center}
\emph{``A Great and Mischievous Pirate''
{\begin{flushright}$\sim$ Henry Morgan\end{flushright}}}
\end{center}
\end{abstract}

%% \category{CR-number}{subcategory}{third-level}
\terms
Languages, JIT, Tracing
\keywords
piracy, rpython, tracing, jit

\section{\emph{Laurencillo}} %% Introduction


There has been tremendous work into JITs and JIT optimizations in recent years.
Most notably, the RPython tracing JIT has come to be a de-facto standard for the
power of a tracing JIT, driving their flagship python interpreter PyPy. In this
work we present an implementation of the Spinless, Tagless G-Machine (herein the
STG Machine) as an interpreter written on top of RPython. 

The original aim of this work was to construct a fast implementation of the
STG machine that might support a reasonable subset of the compilation output
of GHC. To this end, our implementation is faithful to the original semantics
presented by \citet{spj:stgmachine}.

\section{\emph{Tigre}} %% The Spineless, Tagless G-Machine

In this section, we give a brief sketch of the implementation approach we take
to construct the STG Machine in RPython, discussing the general structure of the
implementation. To begin, we sketch the directory structure and describe each file
there:

\begin{itemize}
\item \lstinline{laurens/}
  \begin{itemize}
    \item \lstinline{config.py} -- Configuration file (currently unused)
    \item \lstinline{debug.py}  -- Debug utilities, primarily for printing
                                   machine states during execution
    \item \lstinline{op.py}     -- The operation nodes, which each contain
                                   a \lstinline{step} operation that performs
                                   a machine reduction.
    \item \lstinline{parse.py}  -- The sketch of an input parser (currently 
                                   unstarted)
    \item \lstinline{stg.py}    -- The main loop of the implementation
    \item \lstinline{tests.py}  -- A set of tests written in terms of AST
                                   nodes for the machine; this serves as
                                   an alternative entry point
  \item \lstinline{laurens/ast/}
    \begin{itemize}
      \item \lstinline{ast.py}  -- Contains the AST nodes that make up expressions;
                                   part of \lstinline{op.py} should likely be
                                   refactored into this file
      \item \lstinline{cont.py} -- Continuations that are pushed to the Return Stack 
    \end{itemize}
  \item \lstinline{laurens/data/}
    \begin{itemize}
      \item \lstinline{argstack.py} -- Stack for arguments to functions
      \item \lstinline{closure.py}  -- Data structure to represent closures
      \item \lstinline{config.py}   -- Configuration file (currently unused)
      \item \lstinline{heap.py}     -- Heap structure
      \item \lstinline{heapobj.py}  -- Objects that live on the heap
      \item \lstinline{retstack.py} --
      \item \lstinline{updframe.py} --
      \item \lstinline{updstack.py} -- 
    \end{itemize}
  \end{itemize}
\end{itemize}


\section{\emph{Francesca}} %% RPython

In this section, we discuss a number of intricacies that require careful
consideration during implementation of the STG machine.

\subsection{The Configuration Structure}
We begin by briefly sketching the machine configuration, which contains
six individual pieces:
\[\langle op; as; rs; us; h; \sigma\rangle\]
Unsurprisingly, each component plays a unique role in the machine:
\begin{itemize}
  \item $op$ contains the operation for the reduction to perform;
  \item $as$ contains the argument stack, which maintains arguments
        during applications;
  \item $rs$ contains the return stack, which manages return points
        (primarily ``case'' expressions);
  \item $us$ contain the update stack, used for handling thunk updates;
  \item $h$ contains the heap, where memory addresses map to closures;
  \item $\sigma$ contains the \emph{global environment} of bindings.
\end{itemize}
While \citet{spj:stgmachine} observe that many of these stacks may
be combined during compilation to C as an optimization, we maintain
this configuration explicitly in our implementation:
\begin{code}
class Config(object):
  def __init__(self, code, args, ret,  
                     upd, heap, genv):
    self.code       = code
    self.arg_stack  = args
    self.ret_stack  = ret
    self.upd_stack  = upd
    self.heap       = heap
    self.global_env = genv
\end{code}
We take this approach for two reasons. First, we are working from the
assumption that RPython should be able to properly handle the interpretation
structure of the STG machine, and thus these individual stacks should not
cause critical, performance-inhibiting overhead.
Second, RPython performs immense amounts of type inference and combining
these stacks would require that we carefully package each different node
in a new object to indicate which of the original stacks it lived on,
thus adding another level of indirection that the compiler must 
overcome\footnote{There are other mechanisms to avoid this discussed
in the original work, but they require subtle bit manipulation that
would need to be faked in a higher-level implementation.}. 

This automatic type distinction has a further problem, however: we need
identical classes for each of the \lstinline{arg_stack}, \lstinline{ret_stack},
and \lstinline{upd_stack} so that RPython may infer each to a different,
appropriate type. Indeed, we would suggest that this is one of the largest
hurdles in constructing a project such as this in RPython: there is a non-zero
amount of code duplication required to simply exist in the RPython ecosystem.

\subsection{The Funny Case of Laziness}

While standard lazy languages provide forcing at primitive operations, the
STG machine departs from this explicit approach. Instead, a primitive operation
assumes that its arguments are pre-evaluated, and this became troublesome when
testing them. To further elaborate this point, observe a standard definition
of addition in the STG language:
\begin{code}[language=Haskell]
+ = \ {} \\n {n,m} -> case n of
                        n' -> case m of
                                m' -> #+ n' m'
\end{code}
This snippet is interpreted as: 
\begin{center}
\emph{
``Plus is a lambda with no free variables that
is should not be updated. When applied, it will \lstinline{case} on
\lstinline{n} and then \lstinline{m}, using the forcing position of
\lstinline{case} to ensure that both inputs are evaluated before using the
primitive addition operator \lstinline{\#+} on the result.''
}
\end{center}
While not particularly tied to the implementation in RPython, we found this
insight invaluable during implementation and testing and so we have included
it here. This manifests with the following dispatch for addition in our
implementation:
\begin{code}
elif expr_type is ast.ast.PrimOp:
  if cexp.oper == "+":
    lookups = vals(code.env,
                   global_env, 
                   cexp.atoms)
    x1      = lookups[0]
    x2      = lookups[1]
    assert isinstance(x1,ast.ast.ValAST)
    assert isinstance(x2,ast.ast.ValAST)
    res     = x1.value + x2.value
    config.code = op.ReturnInt(res) 
\end{code}
We first ensure that we are performing addition and then look up the operators
with \lstinline{vals} (which uses the local and global environments for variable
lookup). Next, we ensure that each of the resulting lookups are valid Value
nodes, extract their values, and then construct a \lstinline{ReturnIn} operation
code for the configuration.

\section{\emph{Neptune} n\'{e}e \emph{Fortune}} %% Conclusion / Future Work

%% \section{Appendix Title}
%% 
%% This is the text of the appendix, if you need one.

%% \acks
%% 
%% Acknowledgments, if needed.

% We recommend abbrvnat bibliography style.

\bibliographystyle{abbrvnat}

% The bibliography should be embedded for final submission.

\softraggedright
\bibliography{p}

%%\appendix
%%\input{proof-appendix.tex}

\end{document}
